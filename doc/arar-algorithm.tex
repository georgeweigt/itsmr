\documentclass[12pt]{article}
\usepackage{amsmath}
\parindent=0pt
\pdfinfoomitdate=1
\pdftrailerid{}
\begin{document}

The first step in the ARAR algorithm requires finding $\phi$ that minimizes
\begin{equation*}
\epsilon=\sum_{t=\tau+1}^n[Y_t-\phi Y_{t-\tau}]^2
\end{equation*}

where $\tau$ is a lag.
Note that $\epsilon$ is strictly nonnegative and so is minimized for
\begin{equation*}
\frac{d\epsilon}{d\phi}=0
\end{equation*}

Expand $\epsilon$.
\begin{equation*}
\epsilon=\sum_{t=\tau+1}^nY_t^2-2\phi\sum_{t=\tau+1}^nY_tY_{t-\tau}+\phi^2\sum_{t=\tau+1}^nY_{t-\tau}^2
\end{equation*}

Solve for $\phi$ in
\begin{equation*}
\frac{d\epsilon}{d\phi}=-2\sum_{t=\tau+1}^nY_tY_{t-\tau}+2\phi\sum_{t=\tau+1}^nY_{t-\tau}^2=0
\end{equation*}

to obtain
\begin{equation*}
\phi=\frac{\sum_{t=\tau+1}^nY_tY_{t-\tau}}{\sum_{t=\tau+1}^nY_{t-\tau}^2}
\end{equation*}

In R code
\begin{equation*}
\sum_{t=\tau+1}^nY_tY_{t-\tau}=\text{\tt sum(y[(tau+1):n]*y[1:(n-tau)])}
\end{equation*}

and
\begin{equation*}
\sum_{t=\tau+1}^nY_{t-\tau}^2=\text{\tt sum(y[1:(n-tau)]\^{}2)}
\end{equation*}

\end{document}
